\documentclass[../main.tex]{subfiles}

\begin{document}

Tuning the hyperparameters of each temporal difference learning sub-agent independently when working with a vast network of such sub-agents is unfeasible. Because of this, groups of learning sub-agents are usually tuned together. However, tuning learning sub-agents in groups naturally results in less desirable learning. An alternative approach to this strategy is to adapt the hyperparameters of the learning agents on the fly. While there is an extensive body of work adapting the step-size parameter on the fly \cite{zeiler2012adadelta, kearney2017adapting} there is not yet much work on adapting the other key hyperparameter used in temporal difference methods, the trace-decay parameter \cite{white2016greedy}. Therefore we focus our efforts on evaluating the effectiveness of one such method, the {$\lambda$}-greedy algorithm \cite{white2016greedy} when it is used to learn a horde of general value functions \cite{sutton2011horde}.

We apply the {$\lambda$}-greedy algorithm \cite{white2016greedy} to the task of learning a horde of general value functions \cite{sutton2011horde} defined over the sensorimotor data stream of a robot. We additionally extend the {$\lambda$}-greedy algorithm to make use of a recent, more robust variance estimation technique which estimates the variance of the return directly rather than combining estimates of the first and second moments \cite{sherstan2018directly}. We find that both versions of the {$\lambda$}-greedy algorithms have similar performance to GTD($\lambda$) \cite{sutton2009fast} with the original version of the {$\lambda$}-greedy algorithm performing the best overall. We find that the modified version of the {$\lambda$}-greedy algorithm seems to adjust $\lambda$ in a smoother manner compared to the original {$\lambda$}-greedy algorithm at the cost of worse performance.

The remainder of this report expands on these results with an explanation of the robot used to generate the sensorimotor data stream in Section~\ref{sec:robot}. Afterward, in Section~\ref{sec:domain}, we describe the abstract interpretation of the sensorimotor data stream on which we define our value functions. Then, in Section~\ref{sec:value_functions}, we elaborate on the specific value functions that compose of horde. Following that, in Section~\ref{sec:results}, we describe the results of our experiment. Finally we conclude in Section~\ref{sec:conclusion}.

\end{document}
